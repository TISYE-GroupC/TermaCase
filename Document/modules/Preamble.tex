\usepackage{lastpage}


\newcommand{\HRule}{\rule{\linewidth}{0.8mm}}

%Tekst i fotter
\newcommand{\footerText}{\thepage\xspace /\pageref{LastPage}}

\chapterstyle{hangnum}

\nouppercaseheads
\makepagestyle{mystyle} 

\makeevenhead{mystyle}{}{\\ \leftmark}{} 
\makeoddhead{mystyle}{}{\\ \leftmark}{} 
\makeevenfoot{mystyle}{}{\footerText}{} 
\makeoddfoot{mystyle}{}{\footerText}{} 
\makeatletter
\makepsmarks{mystyle}{% Overskriften på sidehovedet
  \createmark{chapter}{left}{shownumber}{\@chapapp\ }{.\ }} 
\makeatother
\makefootrule{mystyle}{\textwidth}{\normalrulethickness}{0.4pt}
\makeheadrule{mystyle}{\textwidth}{\normalrulethickness}

\makepagestyle{plain}
\makeevenhead{plain}{}{}{}
\makeoddhead{plain}{}{}{}
\makeevenfoot{plain}{}{\footerText}{}
\makeoddfoot{plain}{}{\footerText}{}
\makefootrule{plain}{\textwidth}{\normalrulethickness}{0.4pt}

\pagestyle{mystyle}

%%----------------------------------------------------------------------
%
%%Redefining chapter style
%%\renewcommand\chapterheadstart{\vspace*{\beforechapskip}}
%\renewcommand\chapterheadstart{\vspace*{10pt}}
%\renewcommand\printchaptername{\chapnamefont }%\@chapapp}
%\renewcommand\chapternamenum{\space}
%\renewcommand\printchapternum{\chapnumfont \thechapter}
%\renewcommand\afterchapternum{\space: }%\par\nobreak\vskip \midchapskip}
%\renewcommand\printchapternonum{}
%\renewcommand\printchaptertitle[1]{\chaptitlefont #1}
\setlength{\beforechapskip}{0pt} 
\setlength{\afterchapskip}{0pt} 
%\setlength{\voffset}{0pt} 
\setlength{\headsep}{25pt}
%\setlength{\topmargin}{35pt}
%%\setlength{\headheight}{102pt}
%\setlength{\textheight}{302pt}
\renewcommand\afterchaptertitle{\par\nobreak\vskip \afterchapskip}
%%----------------------------------------------------------------------




%Sidehoved og -fod pakke
%Margin
\usepackage[left=3cm,right=2cm,top=2.5cm,bottom=2cm]{geometry}
\usepackage{lastpage}



%%URL kommandoer og sidetal farve
%%Kaldes med \url{www...}
%\usepackage{color} %Skal også bruges
\usepackage{hyperref}
\hypersetup{ 
	colorlinks	= true, 	% false: boxed links; true: colored links
    urlcolor	= blue,		% color of external links
    linkcolor	= black, 	% color of page numbers
    citecolor	= blue,
}



%Mellemrum mellem linjerne    
\linespread{1.5}


%Seperated files
%--------------------------------------------------
%Opret filer således:
%\documentclass[Navn-på-hovedfil]{subfiles}
%\begin{document}
% Indmad
%\end{document}
%
% I hovedfil inkluderes således:
% \subfile{navn-på-subfil}
%--------------------------------------------------
\usepackage{subfiles}

%Prevent wierd placement of figures
%\usepackage[section]{placeins}

%Standard sti at søge efter billeder
%--------------------------------------------------
%\begin{figure}[hbtp]
%\centering
%\includegraphics[scale=1]{filnavn-for-png}
%\caption{Titel}
%\label{fig:referenceNavn}
%\end{figure}
%--------------------------------------------------
\usepackage{graphicx}
\usepackage{subcaption}
\usepackage{float}
\graphicspath{{../Figures/}}

%Speciel skrift for enkelt linje kode
%--------------------------------------------------
%Udskriver med fonten 'Courier'
%Mere info her: http://tex.stackexchange.com/questions/25249/how-do-i-use-a-particular-font-for-a-small-section-of-text-in-my-document
%Eksempel: Funktionen \code{void Hello()} giver et output
%--------------------------------------------------
\newcommand{\code}[1]{{\fontfamily{pcr}\selectfont #1}}


% Følgende er til koder.
%----------------------------------------------------------
%\begin{lstlisting}[caption=Overskrift på boks, style=Code-C++, label=lst:referenceLabel]
%public void hello(){}
%\end{lstlisting}
%----------------------------------------------------------

%Exstra space
\usepackage{xspace}
%Navn på bokse efterfulgt af \xspace (hvis det skal være mellemrum
%gives det med denne udvidelse. Ellers ingen mellemrum.
\newcommand{\codeTitle}{Kodeudsnit\xspace}

%Pakker der skal bruges til lstlisting
\usepackage{listings}
\usepackage{color}
\usepackage{textcomp}
\definecolor{listinggray}{gray}{0.9}
\definecolor{lbcolor}{rgb}{0.9,0.9,0.9}
\renewcommand{\lstlistingname}{\codeTitle}
\lstdefinestyle{Code}
{
	keywordstyle	= \bfseries\ttfamily\color[rgb]{0,0,1},
	identifierstyle	= \ttfamily,
	commentstyle	= \color[rgb]{0.133,0.545,0.133},
	stringstyle		= \ttfamily\color[rgb]{0.627,0.126,0.941},
	showstringspaces= false,
	basicstyle		= \small,
	numberstyle		= \footnotesize,
%	numbers			= left, % Tal? Udkommenter hvis ikke
	stepnumber		= 2,
	numbersep		= 6pt,
	tabsize			= 2,
	breaklines		= true,
	prebreak 		= \raisebox{0ex}[0ex][0ex]{\ensuremath{\hookleftarrow}},
	breakatwhitespace= false,
%	aboveskip		= {1.5\baselineskip},
  	columns			= fixed,
  	upquote			= true,
  	extendedchars	= true,
 	backgroundcolor = \color{lbcolor},
	lineskip		= 1pt,
%	xleftmargin		= 17pt,
%	framexleftmargin= 17pt,
	framexrightmargin	= 0pt, %6pt
%	framexbottommargin	= 4pt,
}

%Bredde der bruges til indryk
%Den skal være 6 pt mindre
\usepackage{calc}
\newlength{\mywidth}
\setlength{\mywidth}{\textwidth-6pt}


% Forskellige styles for forskellige kodetyper
\usepackage{caption}
\DeclareCaptionFont{white}{\color{white}}
\DeclareCaptionFormat{listing}%
{\colorbox[cmyk]{0.43, 0.35, 0.35,0.35}{\parbox{\mywidth}{\hspace{5pt}#1#2#3}}}
\captionsetup[lstlisting]
{
	format			= listing,
	labelfont		= white,
	textfont		= white, 
	singlelinecheck	= false, 
	width			= \mywidth,
	margin			= 0pt, 
	font			= {bf,footnotesize}
}

\lstdefinestyle{Code-C} {language=C, style=Code}
\lstdefinestyle{Code-Java} {language=Java, style=Code}
\lstdefinestyle{Code-C++} {language=[Visual]C++, style=Code}
\lstdefinestyle{Code-VHDL} {language=VHDL, style=Code}
\lstdefinestyle{Code-Bash} {language=Bash, style=Code}

%Text typesetting
%--------------------------------------------------------
%\usepackage{baskervald}
\usepackage{lmodern}
\usepackage[T1]{fontenc}              
\usepackage[utf8]{inputenc}         
\usepackage[english]{babel}       

\setlength{\parindent}{0pt}
\nonzeroparskip

%\setaftersubsecskip{1sp}
%\setaftersubsubsecskip{1sp}
 


%Dybde på indholdsfortegnelse
%----------------------------------------------------------
%Chapter, section, subsection, subsubsection
%----------------------------------------------------------
\setcounter{secnumdepth}{3}
\setcounter{tocdepth}{3}


%Tables
%----------------------------------------------------------
\usepackage{tabularx}
\usepackage{array}
\usepackage{multirow} 
\usepackage{multicol} 
\usepackage{booktabs}
\usepackage{wrapfig}
\renewcommand{\arraystretch}{1.5}



%Misc
%----------------------------------------------------------
\usepackage{cite}
\usepackage{appendix}
\usepackage{amssymb}
\usepackage{url,ragged2e}
\usepackage{enumerate}
\usepackage{amsmath} %Math bibliotek


\usepackage{longtable}
%\usepackage{enumitem} %Styring af itemize og enumerate



\newcommand{\itoc}{I$^2$C \xspace}

%Software function evironment
%\begin{Function}
%\name{•}
%\para{•}
%\retu{•}
%\comm{•}
%\end{Function}
\newcommand{\class}[1]{\newpage \subsubsection{#1}}
\newenvironment{Function}
{\begin{longtable}{p{0.2\textwidth} p{0.75\textwidth}}}
{\end{longtable}}

\newcommand{\name}[1]{\hline \textbf{Navn: }& \code{#1}}
\newcommand{\para}[1]{\\ \hline \textbf{Parametre: }& \code{#1}}
\newcommand{\retu}[1]{\\ \textbf{Returværdi: }& \code{#1}}
\newcommand{\comm}[1]{\\ \textbf{Funktionalitet: }& \parbox[t]{0.75\textwidth}{#1}}


%UC environment:
%Use as follow:
%\begin{UseCase}
%\UCnumber{0.5}{Test}
%\UCgoal{Goal of Use Case.}
%\UCinit{Use Case initieres of \dots}
%\UCslutSuc{Succes }
%\UCslutUnSuc{Unsucces}
%
%\UCnormFor{Normal happening 1: \\ Do stuff}
%\begin{normFor}
%\item Do this
%\item Do that
%\item[] \textit{However 1}
%\end{normFor}
%
%\UCnormFor{Normal happening 2: \\ Do other stuff}
%\begin{normFor}
%\item Do this
%\item[] \textit{However 2}
%\item Do that
%\end{normFor}
%
%\UCnormFor{Execptions: }
%\begin{normFor}
%\item[] \textit{However 1}
%	\begin{itemize}
%	\item Do this	
%	\end{itemize}
%\item[] \textit{However 2}
%	\begin{itemize}
%	\item Do this
%	\item Then that
%	\end{itemize}
%\end{normFor}
%
%
%\end{UseCase}

\newcommand{\UCnumber}[2]{%
\hline Use Case: #1 & #2%
}
\newcommand{\UCgoal}{\\ \hline Mål: &}
\newcommand{\UCinit}{\\ Initiering: & }
\newcommand{\UCstartb}{\\ Startbetingelser: & }
\newcommand{\UCslutSuc}{\\ Slutbetingelser ved succes: & }
\newcommand{\UCslutUnSuc}{\\ Slutbetingelser ved undtagelser: & }
\newcommand{\UCenum}{noitemsep, leftmargin=*}%, nolistsep}
\newcommand{\UCnormFor}[1]{\\ \hline \parbox[t]{0.25\textwidth}{#1}&}



\newenvironment{UseCase}
{\begin{longtable}{| p{0.25\textwidth} | p{0.70\textwidth} |}}
{\\ \hline \end{longtable}}


\newenvironment{normFor}
{\vspace{-8mm}\begin{enumerate}[noitemsep,nolistsep,leftmargin=*]}
{\end{enumerate}}



%--------------------------------------------------
%\begin{TestCaseIntro}
%\TCprep{Preperation of sorts \\Anything here}
%\TCdesc{Description of what is ablut to happen}
%\end{TestCaseIntro}
%--------------------------------------------------
\newcommand{\TCdescWidth}{0.15 \textwidth}
\newcommand{\TCIntroWidth}{0.8\textwidth}
\newcommand{\TCprep}[1]{\textbf{Preparation:} & \parbox[t]{\TCIntroWidth}{#1}}
\newcommand{\TCdesc}[1]{\\ \textbf{Description:} & \parbox[t]{\TCIntroWidth}{#1}}

\newenvironment{TestCaseIntro}
{\begin{longtable}{p{\TCdescWidth} p{\TCIntroWidth}}}
{\end{longtable}}
%--------------------------------------------------

\newcounter{ctStep}
\setcounter{ctStep}{0}
\newcommand{\step}{%
\addtocounter{ctStep}{1}%
\arabic{ctStep}}


%--------------------------------------------------
%\begin{TestCase}
%\TC
%{Steps}
%{Aktion to take}
%{Expectation}
%{Leave blank}
%\end{TestCase}
%--------------------------------------------------
\newcommand{\TCsteps}{0.05 \textwidth}
\newcommand{\TCexpected}{0.38 \textwidth}
\newcommand{\TCmethod}{0.38 \textwidth}
\newcommand{\TCsucces}{0.08 \textwidth}

\newenvironment{TestCase}
{\begin{longtable}{ p{\TCsteps}  p{\TCexpected} p{\TCmethod} p{\TCsucces} }
\hline \textbf{Step} & \textbf{Method} & \textbf{Expected} & \textbf{Succes} \\ \hline}
{\hline \end{longtable}
\setcounter{ctStep}{0}}

\newcommand{\TC}[2]{\parbox[t]{\TCsteps}{\step} &\parbox[t]{\TCmethod}{#1} &\parbox[t]{\TCexpected}{#2}&  \\}
%--------------------------------------------------