\usepackage{tikz}
\usetikzlibrary{arrows}
\usepackage{datenumber}
\usepackage{xifthen}

% counters for calculating with dates
\newcounter{startdate}
\newcounter{enddate}
\newcounter{tempdate}
\newcounter{dateone}
\newcounter{datetwo}

% 
\newcommand{\startenddiff}[6]{%
\setmydatenumber{startdate}{#1}{#2}{#3}%
\setmydatenumber{enddate}{#4}{#5}{#6}%
\setmydatenumber{tempdate}{#4}{#5}{#6}%
\addtocounter{tempdate}{-\thestartdate}%
}

%
\newcommand{\datediff}[6]%
{   \setmydatenumber{dateone}{#1}{#2}{#3}
    \setmydatenumber{datetwo}{#4}{#5}{#6}
    \addtocounter{dateone}{-\thestartdate}
    \addtocounter{datetwo}{-\thestartdate}
}

%
\newcommand{\drawtimeline}%
{   \setdatebynumber{\thestartdate}
    \pgfmathtruncatemacro{\numberofdays}{\thetempdate}
    \pgfmathsetmacro{\daywidth}{\timelinewidth/\numberofdays}
    \draw[-stealth] (0,0) -- (\timelinewidth,0) -- ++(0.3,0);
    \foreach \x in {0,...,\numberofdays}
    { \ifthenelse{\thedateday = 1}
        { \ifcase\thedatemonth
            \or \xdef\monthname{Jan}
            \or \xdef\monthname{Feb}
            \or \xdef\monthname{Mar}
            \or \xdef\monthname{Apr}
            \or \xdef\monthname{May}
            \or \xdef\monthname{Jun}
            \or \xdef\monthname{Jul}
            \or \xdef\monthname{Aug}
            \or \xdef\monthname{Sep}
            \or \xdef\monthname{Oct}
            \or \xdef\monthname{Nov}
            \or \xdef\monthname{Dec}    
            \else       
            \fi
            \draw (\x*\daywidth,0) -- (\x*\daywidth,0.25) node[right,rotate=60,font=\tiny] {\monthname\ \thedateyear};
        }{}
        \ifthenelse{\equal{\datedayname}{Monday}}
        { \draw (\x*\daywidth,0) -- (\x*\daywidth,-0.05);
        }{}
        \addtocounter{datenumber}{1}
        \setdatebynumber{\thedatenumber}
    }
}

\newcommand{\timeentry}[8][gray]% [options] start date, end date, description
{ \datediff{#2}{#3}{#4}{#5}{#6}{#7}
    \pgfmathtruncatemacro{\numberofdays}{\thetempdate}
    \pgfmathsetmacro{\daywidth}{\timelinewidth/\numberofdays}
    \draw[opacity=0.5,line width=1.5mm,line cap=round,#1] (\thedateone*\daywidth,0) -- (\thedatetwo*\daywidth,0) node[left,rotate=60,pos=0.5] {#8};
}

% Example
% Reference http://tex.stackexchange.com/questions/136882/how-to-use-the-chronology-sty-package
% ===== user's choices =========================
%\startenddiff{2013}{07}{01}{2013}{10}{01}
%\pgfmathsetmacro{\timelinewidth}{10}% will be 0.3cm wider for arrow tip
% ==============================================
%\begin{tikzpicture}
%    \drawtimeline
%    \timeentry[red,text opacity=1]{2013}{08}{14}{2013}{08}{19}{Test}
%    \timeentry[blue]{2013}{07}{08}{2013}{07}{15}{Test 2}
%\end{tikzpicture}
