\documentclass[Main]{subfiles}
\begin{document}
\chapter{Test description}

\section{Test 1}


\begin{TestCaseIntro}
\TCprep{The pod is loaded with a maximal amount of magazines. 
The pod shall be fully assembled. 
The pod does not need to be powered up.}
\TCdesc{The weight of the pod is tested for not exceeding the specified amount.}
\end{TestCaseIntro}

\begin{TestCase}
\TC
{The pod is placed upon a weight.}
{The weight is less than 270 kg.}

\end{TestCase}


\newpage
\section{Test 2}

\begin{TestCaseIntro}
\TCprep{The pod is loaded with live magazines and dummy sensors. 
The pod shall be fully assembled and powered up.
The system shall start in semi-automatic mode.
\\
The test pattern shall be selected.
The test pattern will fire 20 payloads pr second, 2 at a time, 10 forwards and backwards, 10 to each side.
\\
The test bench will simulate the interface of an F-16 being in flight.}
\TCdesc{Tests the pod's communication and threat response.}
\end{TestCaseIntro}

\begin{TestCase}
\TC
{Connect the pod to an electrical test bench.}
{}

\TC
{Request status using the MIL-STD-1553-B databus.}
{System responds with status in less than 5 ms.}


\TC
{The system is set in automatic mode.}
{The system's status is shown on the mission computer in less than 5 ms.}

\TC
{A threat is simulated with dummy sensors.}
{The system transmits audio cues to the pilot.

The system dispenses the test pattern at the required rates automatically.

Updated amount of available payload is made available to the Mission computer}


\end{TestCase}


\newpage
\section{Test 3}


\begin{TestCaseIntro}
\TCprep{The pod must be fully assembled and loaded.
The pod does not need to be powered up.}
\TCdesc{The pod is tested for exposure to acceleration.}
\end{TestCaseIntro}

\begin{TestCase}
\TC
{Mount the pod in an centrifuge}
{}

\TC
{Test for downwards acceleration for up to 11g}
{The structure will remain intact}

\TC
{Test for upwards acceleration for up to 25g}
{The structure will remain intact}

\TC
{Test for forwards acceleration for up to 5g}
{The structure will remain intact}

\TC
{Test for aft acceleration for up to 2.5g}
{The structure will remain intact}

\TC
{Dismount the pod and power it up.}
{All systems powers up and report no failures}
\\
\end{TestCase}


\newpage
\section{Test 4}


\begin{TestCaseIntro}
\TCprep{The pod is loaded with a maximal amount of magazines. 
The pod shall be fully assembled and powered up.}
\TCdesc{The pod's resistance to high temperature.}
\end{TestCaseIntro}

\begin{TestCase}
\TC
{Place the pod in a temperature test chamber.}
{}


\TC
{Heat the chamber to 102$^\circ$ Celsius for 25 minutes.}
{}


\TC
{Remove from test chamber and let it cool for 10 minutes.}
{Self-test reports no failures.}

\TC
{Repeat step 1 and heat the chamber to 151$^\circ$ Celsius for 3 minutes.} 
{}

\TC
{Remove from test chamber and let it cool for 10 minutes.}
{Self-test reports no failures.}

\\
\end{TestCase}


\newpage
\section{Test 5}

\begin{TestCaseIntro}
\TCprep{The plane must remain on the ground. The system must be reset.}
\TCdesc{Integration test of the system on the ground.}
\end{TestCaseIntro}

\begin{TestCase}
\TC
{The pod is mounted under the left wing}
{The pod remains mounted under the left wing}

\TC
{Inspect whether the pod obstructs any current weapons.}
{The pod does not obstruct any current weapon systems.}

\TC
{Eight magazines shall be loaded into the pod.}
{}

\TC
{Power to the pod is established to the pod from the cockpit.}
{Status on the mission computer shows loaded magazines ready for use and system status.}

\TC
{A dispensing program is loaded into the Cockpit Unit.}
{The programs are stored.}

\TC
{The system is placed in manual-mode and the "fire"-button is activated.}
{The system will not fire.}
\\

\end{TestCase}


\end{document}